% Options for packages loaded elsewhere
\PassOptionsToPackage{unicode}{hyperref}
\PassOptionsToPackage{hyphens}{url}
\PassOptionsToPackage{dvipsnames,svgnames,x11names}{xcolor}
%
\documentclass[
  letterpaper,
  DIV=11,
  numbers=noendperiod]{scrartcl}

\usepackage{amsmath,amssymb}
\usepackage{lmodern}
\usepackage{iftex}
\ifPDFTeX
  \usepackage[T1]{fontenc}
  \usepackage[utf8]{inputenc}
  \usepackage{textcomp} % provide euro and other symbols
\else % if luatex or xetex
  \usepackage{unicode-math}
  \defaultfontfeatures{Scale=MatchLowercase}
  \defaultfontfeatures[\rmfamily]{Ligatures=TeX,Scale=1}
\fi
% Use upquote if available, for straight quotes in verbatim environments
\IfFileExists{upquote.sty}{\usepackage{upquote}}{}
\IfFileExists{microtype.sty}{% use microtype if available
  \usepackage[]{microtype}
  \UseMicrotypeSet[protrusion]{basicmath} % disable protrusion for tt fonts
}{}
\makeatletter
\@ifundefined{KOMAClassName}{% if non-KOMA class
  \IfFileExists{parskip.sty}{%
    \usepackage{parskip}
  }{% else
    \setlength{\parindent}{0pt}
    \setlength{\parskip}{6pt plus 2pt minus 1pt}}
}{% if KOMA class
  \KOMAoptions{parskip=half}}
\makeatother
\usepackage{xcolor}
\setlength{\emergencystretch}{3em} % prevent overfull lines
\setcounter{secnumdepth}{5}
% Make \paragraph and \subparagraph free-standing
\ifx\paragraph\undefined\else
  \let\oldparagraph\paragraph
  \renewcommand{\paragraph}[1]{\oldparagraph{#1}\mbox{}}
\fi
\ifx\subparagraph\undefined\else
  \let\oldsubparagraph\subparagraph
  \renewcommand{\subparagraph}[1]{\oldsubparagraph{#1}\mbox{}}
\fi


\providecommand{\tightlist}{%
  \setlength{\itemsep}{0pt}\setlength{\parskip}{0pt}}\usepackage{longtable,booktabs,array}
\usepackage{calc} % for calculating minipage widths
% Correct order of tables after \paragraph or \subparagraph
\usepackage{etoolbox}
\makeatletter
\patchcmd\longtable{\par}{\if@noskipsec\mbox{}\fi\par}{}{}
\makeatother
% Allow footnotes in longtable head/foot
\IfFileExists{footnotehyper.sty}{\usepackage{footnotehyper}}{\usepackage{footnote}}
\makesavenoteenv{longtable}
\usepackage{graphicx}
\makeatletter
\def\maxwidth{\ifdim\Gin@nat@width>\linewidth\linewidth\else\Gin@nat@width\fi}
\def\maxheight{\ifdim\Gin@nat@height>\textheight\textheight\else\Gin@nat@height\fi}
\makeatother
% Scale images if necessary, so that they will not overflow the page
% margins by default, and it is still possible to overwrite the defaults
% using explicit options in \includegraphics[width, height, ...]{}
\setkeys{Gin}{width=\maxwidth,height=\maxheight,keepaspectratio}
% Set default figure placement to htbp
\makeatletter
\def\fps@figure{htbp}
\makeatother

\KOMAoption{captions}{tableheading}
\makeatletter
\makeatother
\makeatletter
\makeatother
\makeatletter
\@ifpackageloaded{caption}{}{\usepackage{caption}}
\AtBeginDocument{%
\ifdefined\contentsname
  \renewcommand*\contentsname{Table of contents}
\else
  \newcommand\contentsname{Table of contents}
\fi
\ifdefined\listfigurename
  \renewcommand*\listfigurename{List of Figures}
\else
  \newcommand\listfigurename{List of Figures}
\fi
\ifdefined\listtablename
  \renewcommand*\listtablename{List of Tables}
\else
  \newcommand\listtablename{List of Tables}
\fi
\ifdefined\figurename
  \renewcommand*\figurename{Figure}
\else
  \newcommand\figurename{Figure}
\fi
\ifdefined\tablename
  \renewcommand*\tablename{Table}
\else
  \newcommand\tablename{Table}
\fi
}
\@ifpackageloaded{float}{}{\usepackage{float}}
\floatstyle{ruled}
\@ifundefined{c@chapter}{\newfloat{codelisting}{h}{lop}}{\newfloat{codelisting}{h}{lop}[chapter]}
\floatname{codelisting}{Listing}
\newcommand*\listoflistings{\listof{codelisting}{List of Listings}}
\makeatother
\makeatletter
\@ifpackageloaded{caption}{}{\usepackage{caption}}
\@ifpackageloaded{subcaption}{}{\usepackage{subcaption}}
\makeatother
\makeatletter
\@ifpackageloaded{tcolorbox}{}{\usepackage[many]{tcolorbox}}
\makeatother
\makeatletter
\@ifundefined{shadecolor}{\definecolor{shadecolor}{rgb}{.97, .97, .97}}
\makeatother
\makeatletter
\makeatother
\ifLuaTeX
  \usepackage{selnolig}  % disable illegal ligatures
\fi
\IfFileExists{bookmark.sty}{\usepackage{bookmark}}{\usepackage{hyperref}}
\IfFileExists{xurl.sty}{\usepackage{xurl}}{} % add URL line breaks if available
\urlstyle{same} % disable monospaced font for URLs
\hypersetup{
  pdftitle={Value\_Chain\_Visualization},
  colorlinks=true,
  linkcolor={blue},
  filecolor={Maroon},
  citecolor={Blue},
  urlcolor={Blue},
  pdfcreator={LaTeX via pandoc}}

\title{Value\_Chain\_Visualization}
\author{}
\date{}

\begin{document}
\maketitle
\ifdefined\Shaded\renewenvironment{Shaded}{\begin{tcolorbox}[borderline west={3pt}{0pt}{shadecolor}, frame hidden, sharp corners, breakable, interior hidden, boxrule=0pt, enhanced]}{\end{tcolorbox}}\fi

\renewcommand*\contentsname{Table of contents}
{
\hypersetup{linkcolor=}
\setcounter{tocdepth}{3}
\tableofcontents
}
\hypertarget{purpose-and-objective}{%
\section{Purpose and Objective}\label{purpose-and-objective}}

Resource-Product Value chain analysis and visualization using I-O
accounts as a data source.

\hypertarget{national-income-and-product-accounts}{%
\subsection{1. National Income and Product
Accounts}\label{national-income-and-product-accounts}}

National Income and Product identities: \[
\begin{align}
  v = c+i+g+(e-m)   && \text{(National Income = National Product)}\\
  m+v = c+i+g+e     && \text{(GNI + imports = GNP)}\\
  m+v = f     && \text{(GNP = Total Final Demand = f)}\\
\end{align}
\]

Add Intermediate output variable: \[
\begin{align}
  m+v = x+f     && \text{(x=intermediate output; (x+f) and (m+v)= total output)}\\
\end{align}
\]

\hypertarget{i-o-accounts}{%
\subsection{2. I-O Accounts}\label{i-o-accounts}}

\hypertarget{total-output-total-outlay}{%
\subsubsection{2.1 Total output = total
outlay:}\label{total-output-total-outlay}}

\[
\begin{align}
    X + f = x && \text{(Intermediate Demand + Final Demand = Total Output)}\\
    V + M = q && \text{(Value Added + Imports = Total Outlay)}\\
\end{align}
\]

\hypertarget{direct-input-coefficients}{%
\subsubsection{2,2 Direct input
coefficients}\label{direct-input-coefficients}}

\[
\begin{align}
    \hat{q}^{-1}X = _{i}A && \text{(Inter-industry Direct Input Coefficients matrix iA)}\\
    \hat{q}^{-1}V = _{v}A && \text{(Value Added Direct Input Coefficients matrix vA)}\\
    \hat{q}^{-1}M = _{m}A && \text{(Import Direct Input Coefficients matrix mA)}\\
\end{align}
\]

\hypertarget{inter-industry-adjacency-matrix}{%
\subsubsection{2.3 Inter-Industry Adjacency
Matrix}\label{inter-industry-adjacency-matrix}}

Derive adjacency matrix \textbf{\emph{B}} from the
\textsubscript{i}\textbf{\emph{A}} matrix. Successive powers of
\textbf{\emph{B\textsuperscript{n}}} will identify successive paths of
step-length \textbf{\emph{n}}. Assume industry \textbf{\emph{1}} is
\emph{Grow}, industry \textbf{\emph{2}} is \emph{Harvest}, industry
\textbf{\emph{3}} is \emph{Saw Mill}, industry \textbf{\emph{4}} is
\emph{Residuals}, industry \textbf{\emph{5}} is \emph{Bio-Power}, and
industry \textbf{\emph{6}} is \emph{Other}. Assume the commodity
produced by industry \textbf{\emph{1}} is \emph{Stumpage}
(\emph{m\textsuperscript{3}} of solid wood), the commodity produced by
industry \textbf{\emph{2}} is \emph{Logs} (containing
\emph{m\textsuperscript{3}} of solid wood), the commodity produced by
industry \textbf{\emph{3}} is \emph{Dimension Lumber} (containing
\emph{m\textsuperscript{3}} of solid wood), the commodity produced by
industry \textbf{\emph{4}} is \emph{Sawdust Residuals} (containing
\emph{m\textsuperscript{3}} of solid wood), the commodity produced by
industry \textbf{\emph{5}} is \emph{Electricity} (generated by using
\emph{m\textsuperscript{3}} of solid wood \emph{Sawdust Residuals} as
fuel), and commodity produced by industry \textbf{\emph{6}} is
\emph{Other}.

Matrix \textsubscript{i}A:

\[\mathbf{A} = \left[\begin{array}
{rrr}
a_{11} & a_{12} & a_{13} & a_{14} & a_{15} & a_{16} \\
a_{21} & a_{22} & a_{23} & a_{24} & a_{25} & a_{26} \\
a_{31} & a_{32} & a_{33} & a_{34} & a_{35} & a_{36} \\
a_{41} & a_{42} & a_{43} & a_{44} & a_{45} & a_{46} \\
a_{51} & a_{52} & a_{53} & a_{54} & a_{55} & a_{56} \\
a_{61} & a_{62} & a_{63} & a_{64} & a_{65} & a_{66} \\
\end{array}\right]
\]

Matrix \textsubscript{i}A non-zero elements:

\[\mathbf{A} = \left[\begin{array}
{rrr}
a_{11} & a_{12} & 0 & 0 & 0 & 0 \\
0 & a_{22} & a_{23} & 0 & 0 & 0 \\
0 & 0 & a_{33} & a_{34} & 0 & 0 \\
0 & 0 & 0 & a_{44} & a_{45} & 0 \\
0 & 0 & 0 & 0 & 0 & 0 \\
0 & 0 & 0 & 0 & 0 & 0 \\
\end{array}\right]
\]

Adjacency Matrix B\textsuperscript{1}:

\[\mathbf{B} = \left[\begin{array}
{rrr}
1 & 1 & 0 & 0 & 0 & 0 \\
0 & 1 & 1 & 0 & 0 & 0 \\
0 & 0 & 1 & 1 & 0 & 0 \\
0 & 0 & 0 & 1 & 1 & 0 \\
0 & 0 & 0 & 0 & 0 & 0 \\
0 & 0 & 0 & 0 & 0 & 0 \\
\end{array}\right]
\]

Adjacency Matrix B\textsuperscript{4}:

\[\mathbf{B} = \left[\begin{array}
{rrr}
1 & 4 & 6 & 4 & 1 & 0 \\
0 & 1 & 4 & 5 & 2 & 0 \\
0 & 0 & 1 & 2 & 1 & 0 \\
0 & 0 & 0 & 1 & 1 & 0 \\
0 & 0 & 0 & 0 & 0 & 0 \\
0 & 0 & 0 & 0 & 0 & 0 \\
\end{array}\right]
\]

\hypertarget{leontief-i-o-accounts}{%
\subsection{3 Leontief I-O Accounts:}\label{leontief-i-o-accounts}}

\[
\begin{align}
    _{i}Ax+f=x && \text{(Substitute iA Direct Interindustry Input Coefficients matrix)}\\
        f=(I-_{i}A)x && \text{(Rearrange)}\\
    x = (I-_{i}A)^{-1}f && \text{(Leontief I-O Accounts transformation)}\\
    x = Zf && \text{(Z relates industry input vectors to final products)}\\
\end{align}
\]

\hypertarget{column-space-expansion-of-total-output-vector}{%
\subsubsection{3.1 Column Space Expansion of Total Output
Vector}\label{column-space-expansion-of-total-output-vector}}

Column space expansion of total output vector x:

\[
\begin{align}
    x = Zf && \text{(Leontief transformation function)}\\
    O = Z\hat{f} && \text{(f-hat is a diagonal matrix of final demands; O is the column space of x)}\\
\end{align}\]

Matrix O: \[\mathbf{O} = \left[\begin{array}
{rrr}
o_{11} & o_{12} & o_{13} & o_{14} & o_{15} & o_{16} \\
o_{21} & o_{22} & o_{23} & o_{24} & o_{25} & o_{26} \\
o_{31} & o_{32} & o_{33} & o_{34} & o_{35} & o_{36} \\
o_{41} & o_{42} & o_{43} & o_{44} & o_{45} & o_{46} \\
o_{51} & o_{52} & o_{53} & o_{54} & o_{55} & o_{56} \\
o_{61} & o_{62} & o_{63} & o_{64} & o_{65} & o_{66} \\
\end{array}\right]
\]

\hypertarget{column-space-expansion-of-total-output-vector-o5}{%
\subsubsection{\texorpdfstring{3.1 Column Space Expansion of Total
Output Vector
o\textsubscript{5}}{3.1 Column Space Expansion of Total Output Vector o5}}\label{column-space-expansion-of-total-output-vector-o5}}

Vector o\textsubscript{5}:

\[\mathbf{o} = \left[\begin{array}
{rrr}
o_{15} \\
o_{25} \\
o_{35} \\
o_{45} \\
o_{55} \\
o_{65} \\
\end{array}\right]
\]

Column space expansion of output vector o\textsubscript{5}:

\[
\begin{align}
    o = Zf && \text{(Leontief transformation function)}\\
    O = Z\hat{f} && \text{(f-hat is a diagonal matrix of intermediate demands by Biopwr; O is the column space of o)}\\
\end{align}
\]

Matrix O: \[\mathbf{O} = \left[\begin{array}
{rrr}
o_{11} & o_{12} & o_{13} & o_{14} & o_{15} & o_{16} \\
o_{21} & o_{22} & o_{23} & o_{24} & o_{25} & o_{26} \\
o_{31} & o_{32} & o_{33} & o_{34} & o_{35} & o_{36} \\
o_{41} & o_{42} & o_{43} & o_{44} & o_{45} & o_{46} \\
o_{51} & o_{52} & o_{53} & o_{54} & o_{55} & o_{56} \\
o_{61} & o_{62} & o_{63} & o_{64} & o_{65} & o_{66} \\
\end{array}\right]
\]



\end{document}
